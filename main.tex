\documentclass[17pt, a1paper, landscape, margin=0mm, innermargin=10mm,
blockverticalspace=15mm, colspace=10mm, subcolspace=8mm]{tikzposter}
\usepackage{polyglossia} %% загружает пакет многоязыковой вёрстки
\setdefaultlanguage{russian}
\setotherlanguage{english} 
\defaultfontfeatures{Ligatures={TeX},Renderer=Basic} %% свойства шрифтов по умолчанию. Для XeTeX опцию Renderer=Basic можно не указывать, она необходима для LuaTeX
%\setmainfont[Ligatures={TeX,Historic}]{CMU Serif} %% задаёт основной шрифт документа
%\setsansfont{CMU Sans Serif}
%\setsansfont{Linux Libertine O} %% задаёт шрифт без засечек
%\setmonofont{CMU Typewriter Text} %% задаёт моноширинный шрифт
\usepackage{libertine}
\usepackage[scaled=0.92]{inconsolata}
\usepackage[libertine]{newtxmath}
\usepackage{wrapfig}
\usepackage{lipsum}
\newfontfamily{\cyrillicfonttt}{Courier New}
\usepackage{color}

\usepackage{multicol}
\usepackage{lipsum}
\usepackage{mwe}

\title{Визуализация данных по смертности в России за 1959-2014 гг.}
\author{Бакуева Дженнет, Ведерникова Анастасия, Кислик Мария}
\date{\today}
\institute{Факультет экономических наук НИУ ВШЭ}
 
\usetheme{Board}

\colorlet{backgroundcolor}{pink}
\colorlet{framecolor}{black}
\begin{document}




\maketitle
\begin{columns}
    \column{0.99}
    \block{Графики}{
    \begin{multicols}{3}[\columnsep=2cm]
\includegraphics[scale=1.7]{rus_total.png}

\columnbreak
\includegraphics[scale=1.7]{rus_rel.png}
\columnbreak
\includegraphics[scale=1.7]{rus_exp.png}
\end{multicols}
 }
\end{columns}

\begin{columns}
    \column{0.34}
    \block{Общая смертность}{
    
    % \qquad Основная цель данного проекта — визуализировать массив данных, содержащий информацию о смертности в России за период с 1959 по 2014. В качестве базы данных используется информация из источника www.mortality.org. 
   
    \quad Первый график демонстрирует суммарное количество смертей мужчин и женщин по возрастам (вертикальная ось) и годам (горизонтальная ось). Так как просто число изобразить сложно, был применён следующий приём. Отранжируем суммарное количество смертей и разделим его на $100$ равных страт. Принадлежность количества смертей конкретного возраста в конкретный год к определённой страте и изображена на Графике 1. Например, в 2014 году суммарно умерло $18105$ человек в возрасте $78$ лет, что соответствует $69$-ой страте, тогда как в возрасте $82$ лет умерло уже $36384$ человека, что соответствует $89$-ой страте. Таким образом, чем светлее цвет, тем более высокая страта была присвоена и тем выше была смертность в конкретной точке.
    
      \quad Тёмные полосы, проходящие через весь график, отражают влияние Великой Отечественной войны. Самая тёмная полоса отображает поколение, родившееся в начале сороковых. В связи с высокой детской смертностью и низкой рождаемостью количество смертей этого поколения в последующие годы ниже, так как его существенная часть погибла в ходе войны. Также чуть более светлыми полосами проходят поколения, которым было $10-30$ лет к началу войны. На них также пришлись существенные потери во время войны, и, следовательно, последующая смертность ниже. 
    
      \quad Тонкая светлая полоска в основании графика отражает младенческую смертность, которая с прогрессом здравоохранения заметно снизилась. Также снизилась детская и юношеская смертность в возрасте $10-20$ лет.
    
      \quad Концентрация светлых участков в верхнем правом углу говорит о том, что большее количество людей стало доживать до преклонного возраста. Это также свидетельствует о прогрессе здравоохранения.

    }    
        
     \column{0.25}
    \block{Отношение мужской смертности к женской}{
    
   
    
    \quad Второй график отображает отношение количества умерших мужчин к умершим женщинам для каждого года и возраста. Аналогично предыдущему графику, показатель был разделён на $100$ страт. 
    
      \quad Тёмный треугольник в левой части графика также может быть объяснён войной. Большое количество мужчин погибло на войне и соотношение сильно упало начиная с мужчин, которым на момент начала войны было от $10$ лет.
    
      \quad Ещё одним паттерном является повышенная мужская смертность в возрасте до $30$ лет на протяжении всего наблюдаемого периода. Однако эта тенденция в последние годы стала постепенно ослабевать.
    
      \quad Широкая тёмная полоса в верхней части графика говорит о том, что в последние годы мужчины почти не умирают в возрасте более $90$ лет. Скорее всего это связано со снизившейся продолжительностью жизни мужчин.

    }    
        
    
  \column{0.41}
    \block{Количество подверженных риску смерти}{
    
  
    
    
    \quad Третий график позволяет увидеть суммарное количество мужчин и женщин, подверженных риску смерти для каждого года и возраста. Как и ранее, значение показателя было отранжировано и разбито на $100$ страт.
    
      \quad Светлая полоса сквозь весь график вновь отображает поколение, которое появилось на свет в начале войны либо сразу после неё. Эта группа наиболее подвержена риску смерти, так как детское здоровье куда сильнее подрывается голодом, холодом, потерей близких и прочими невзгодами войны. Дети, которым не хватало еды для полноценного развития организма, априори находятся в группе риска. 
    
      \quad Более тёмные полосы выше связаны уже с ветеранами войны, которые с большой долей вероятности могли иметь ранения во время боевых действий или приобрести в тяжёлых условиях хронические болезни (например, туберкулёз или застуженные почки). Подобные обстоятельства помещают людей в группу риска.
    
      \quad Полосы, связанные с людьми, которым к $1959$ было в районе $50$ лет, также могут быть ветеранами, или их детство вполне могло проходить во время первой мировой войны. Эти факторы также положительно отражаются на риске смерти. 
    
      \quad Аналогично можно выделить ещё две группы, чуть выделяющиеся. Первая - это поколение шестидесятых-семидесятых. Довольно сложно связать это с какими-либо значимыми событиями. Вторая - это поколения "лихих" девяностых. Здесь уже всё более-менее логично. Падение уровня жизни, голод, бедность, а также криминогенная обстановка помещают эти поколения в группу повышенного риска смерти. 
    
    
    }    
    \note[
        targetoffsetx=1.5cm, 
        targetoffsety=-12.5cm, 
        width=0.2\linewidth
        ]
        { \texttt{Источник данных: www.mortality.org}}
    

    
\end{columns}
\end{document}