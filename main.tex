\documentclass[20pt, a1paper, landscape]{tikzposter}
\usepackage{polyglossia} %% загружает пакет многоязыковой вёрстки
\setdefaultlanguage{russian}
\setotherlanguage{english} 
\defaultfontfeatures{Ligatures={TeX},Renderer=Basic} %% свойства шрифтов по умолчанию. Для XeTeX опцию Renderer=Basic можно не указывать, она необходима для LuaTeX
%\setmainfont[Ligatures={TeX,Historic}]{CMU Serif} %% задаёт основной шрифт документа
%\setsansfont{CMU Sans Serif}
%\setsansfont{Linux Libertine O} %% задаёт шрифт без засечек
%\setmonofont{CMU Typewriter Text} %% задаёт моноширинный шрифт
\usepackage{libertine}
\usepackage[scaled=0.92]{inconsolata}
\usepackage[libertine]{newtxmath}
\usepackage{wrapfig}
\usepackage{lipsum}
\newfontfamily{\cyrillicfonttt}{Courier New}
\usepackage{color}


\title{Визуализация данных по смертности в России за 1959-2014 гг.}
\author{Бакуева Дженнет, Ведерникова Анастасия, Кислик Мария}
\date{\today}
\institute{Факультет экономических наук НИУ ВШЭ}
 
\usetheme{Board}

\colorlet{backgroundcolor}{pink}
\colorlet{framecolor}{black}
\begin{document}




\maketitle

\begin{columns}
    \column{0.225}
    \block{Описание данных}{
    \qquad Основная цель данного проекта — визуализировать массив данных, содержащий информацию о смертности в России за период с 1959 по 2014. В качестве базы данных используется информация из источника www.mortality.org. 
   
    \qquad Данные содержат информацию о смертности (оценочное количество умерших) по полу, возрасту и году. Беря самое маленькое значение из базы данных за 0 и самое большое за 100, мы демонстрируем цветами на графике отнормированное число смертей в каждой точке. 
        
    \qquad График 1 показывает число смертей по возрастам в определенном возрасте (вертикальная ось) в конкретный год (горизонтальная ось). Оно было пронормировано к 100 следующим образом: 0 на графике означает наименьшее количество смертей, 25 — число умерших в конкретный год в конкретном возрасте составляет менее четверти от общего значения смертей по выборке из всех наблюдений, 50 — медианное количество смертей, и т.д. по возрастанию, а 100 — максимальное число умерших в выборке.
    
    \qquad График 2 демонстрирует отношение количества умерших мужчин к умершим женщинам в одном возрасте в один год. Чем светлее цвет, тем больше мужчин умерло на 1 женщину.
    
    \qquad В графике 3 рассматривается подверженность населения риску смерти в определенном возрасте.
        
    \qquad Визуализация данных позволила наглядно продемонстрировать тенденции изменения показателей смертности населения.
    }    
        
    \column{0.48}
    \block{Графики}{

\centerline{\begin{minipage}[h]{0.9\linewidth}
\center{\includegraphics[width=1\linewidth]{rus_total.png}} График 1. Общая смертность по возрастам \\
\end{minipage}}
\vfill
\begin{minipage}[h]{0.49\linewidth}
\center{\includegraphics[width=1\linewidth]{rus_rel.png}} \\ График 2. Отношение мужской смертности к женской по возрастам
\end{minipage}
\hfill
\begin{minipage}[h]{0.49\linewidth}
\center{\includegraphics[width=1\linewidth]{rus_exp.png}} График 3. Подверженность риску
смерти по возрастам \\
\end{minipage}
}
        
        \column{0.3}
    
     \block{Выводы}{
     \qquad\textit{Общая смертность по возрастам} 

\qquad График показывает поколение времён Второй Мировой войны темным участком (низкий перцентиль смертности) в возрастном интервале от 20 до 45 лет в послевоенный период. Это объясняется тем, что многие погибли на войне, и поэтому к 1960-м годам относительная смертность была невысокой. На графике заметна тенденция к снижению детской смертности. Например, в 2000 году значение детской смертности в возрасте 7 лет на графике обозначено тёмно-фиолетовым цветом (значение около 25), что говорит о следующем: в данный год число умерших в 7-летнем возрасте невелико, и находится в четверти самых маленьких значений среди всех наблюдений выборки. Концентрация светлых участков в верхней части графика в наши дни говорит, что все больше людей доживает до преклонного возраста, и продолжительность жизни растет. Заметно снижение смертности в среднем возрасте. 
    ~\
   
       \qquad\textit{Отношение мужской смертности к женской по возрастам}

\qquad В левой части графика темная полоса (малое количество мужчин, умерших на одну женщину) может быть объяснено тем, что большое количество мужчин погибло на войне. Также заметна тенденция превышающей мужской смертности в возрасте до 30 лет, однако данная тенденция снизилась в последние годы. Темная полоса сверху говорит о том, что мужчин, доживших до возраста выше 90 лет, практически нет.

  \qquad\textit{ Подверженность риску смерти по возрастам}
  
\qquad На данном графике чередование полос говорит о смене поколений. Можно заметить, что с 1994 года снижается риск смерти новых поколений. Предположительно, это связано с улучшением качества медицины и жизни в целом. В 1984 году в возрасте 25 лет риск умереть составлял 100 (чёрный цвет, т.е. достигались самые большие значения выборки из предоставленной базы данных). Получается, что в 1984 году подверженность риску смерти у 25-летних граждан была максимальной. В последние годы на графике заметно просветление — это знак снижения склонности к риску смерти.

        }
    
    
\end{columns}
\end{document}